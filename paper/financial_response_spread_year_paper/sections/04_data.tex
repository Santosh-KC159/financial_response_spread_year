\section{Data set}\label{sec:data}

In a modern financial market, there is a double continuous auction. To find
possible buyers and sellers in the market, agents can place different types of
orders to buy or to sell a given number of shares, that can be grouped into
two categories: market orders and limit orders.

Market orders will go into market to execute at the best available buy or sell
price. Limit orders allow to set a maximum purchase price for a buy order, or a
minimum sale price for a sell order. If the market does not reach the limit
price, the order will not be executed
\cite{large_prices_changes,predictive_pow,intro_market_micro,stat_theory}.

Limit orders often fail to result in an immediate transaction, and are stored
in a queue called the limit order book
\cite{stat_prop,predictive_pow,intro_market_micro,prop_order_book}. An order
book is an electronic list of buy and sell orders for a specific security or
financial instrument organized by price level. An order book lists the number
of shares being bid or offered at each price point. It also identifies the
market participants behind the buy and sell orders, although some choose to
remain anonymous. The order book is visible for all traders and its main
purpose is to ensure that all traders have the information about what is
offered on the market. The order book is the ultimate microscopic level of
description of financial markets.

Buy limit orders are called ``bids", and sell limit orders are called ``asks".
At any given time there is a best (lowest) offer to sell with price
$a\left(t\right)$, and a best (highest) bid to buy with price $b\left(t\right)$
\cite{subtle_nature,account_spread,limit_ord_spread,prop_order_book,stat_theory}.
The price gap between them is called the spread
$s\left(t\right) = a\left(t\right)-b\left(t\right)$
\cite{subtle_nature,market_digest,Bouchaud_2004,account_spread,large_prices_changes,stat_theory}.
Spreads are significantly positively related to price and significantly
negatively related to trading volume. Companies with more liquidity tend to
have lower spreads
\cite{components_spread_tokyo,effects_spread,account_spread,components_spread}.

In this study, we analyzed trades and quotes (TAQ) data from the NASDAQ stock
market. We selected NASDAQ because it is an electronic exchange where stocks
are traded through an automated network of computers instead of a trading
floor, which makes trading more efficient, fast and accurate. Furthermore,
NASDAQ is the second largest stock exchange based on market capitalization
in the world.

In the TAQ data set, there are two data files for each stock. One gives the
list of all successive quotes. Thus, we have the best bid price, best ask
price, available volume and the time stamp accurate to the second. The other
data file is the list of all successive trades, with the traded price, traded
volume and time stamp accurate to the second. Despite the one second accuracy
of the time stamps, in both files more than one quote or trade may be recorded
in the same second.

Due to the the time stamp accuracy, it is not possible to match each trade with
the directly preceding quote. Hence, we cannot determine the trade sign by
comparing the traded price and the preceding midpoint price
\cite{Wang_2016_cross}. In this case we need to do a preprocessing of the data
to relate the midpoint prices with the trade signs in trade time scale and in
physical time scale.

To analyze the response functions across different liquid stocks in Sects.
\ref{sec:response_functions_imp}, \ref{sec:time_shift} and \ref{sec:short_long},
we select the six companies with the largest average market capitalization
(AMC) (Alphabet Inc., Mastercard Inc., CME Group Inc., Goldman Sachs Group
Inc., Transocean Ltd. and Apache Corp.) in three economic sectors (information
technology, financials and energy) of the S\&P index in 2008. Table
\ref{tab:companies} shows the companies analyzed with their corresponding
symbol and sector, and three average values for a year.

\begin{table*}[htbp]
\begin{threeparttable}
\caption{Analyzed companies.}
\begin{tabular*}{\textwidth}{c @{\extracolsep{\fill}} ccccc}
\toprule
\bf{Company} & \bf{Symbol} & \bf{Sector} & \bf{Quotes}\tnote{1} &
\bf{Trades}\tnote{2} & \bf{Spread}\tnote{3}\tabularnewline
\midrule
Alphabet Inc. & GOOG & Information Technology (IT) & $164489$ & $19029$ &
$0.40\$$\tabularnewline
Mastercard Inc. & MA & Information Technology (IT) & $98909$ & $6977$ &
$0.38\$$\tabularnewline
CME Group Inc. & CME & Financials (F) & $98188$ & $3032$ &
$1.08\$$\tabularnewline
Goldman Sachs Group Inc. & GS & Financials (F) & $160470$ & $26227$ &
$0.11\$$\tabularnewline
Transocean Ltd. & RIG & Energy (E) & $107092$ & $11641$ &
$0.12\$$\tabularnewline
Apache Corp. & APA & Energy (E) & $103074$ & $8889$ & $0.13\$$\tabularnewline
\bottomrule
\end{tabular*}
\label{tab:companies}
\begin{tablenotes}\footnotesize
\item[1] Average number of quotes from 9:40:00 to 15:50:00 New York time during
 2008.
\item[2] Average number of trades from 9:40:00 to 15:50:00 New York time during
 2008.
\item[3] Average spread from 9:40:00 to 15:50:00 New York time during 2008.
\end{tablenotes}
\end{threeparttable}
\end{table*}

To analyze the spread impact in response functions (Sect.
\ref{sec:spread_impact}), we select 524 stocks in the NASDAQ stock market for
the year 2008. The selected stocks are listed in Appendix
\ref{app:spread_impact}.

In order to avoid overnight effects and any artifact due to the opening and
closing of the market, we systematically discarded the first ten and the last
ten minutes of trading in a given day
\cite{Bouchaud_2004,large_prices_changes,spread_changes_affect,Wang_2016_cross}.
Therefore, we only consider trades of the same day from 9:40:00 to 15:50:00
New York local time. We will refer to this interval of time as the ``market
time". The year period 2008 corresponds to 253 bussiness days.
