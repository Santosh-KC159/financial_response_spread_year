\section{Introduction}\label{sec:introduction}

Financial markets use order books to list the number of shares bid or asked at
each price. An order book is an electronic list of buy and sell orders for a
specific security or financial instrument organized by price level, agents can
place different types of instructions (orders).

In general, the dynamics of the prices follow a random walk. There are two
extreme models that can describe this behavior: the Efficient Market Hypothesis
(EMH) and the Zero Intelligence Trading (ZIT). The EMH states that all
available information is included in the price and price changes can only be
the result of unanticipated news, which by definition are totally unpredictable
\cite{subtle_nature,Bouchaud_2004,EMH_lillo,stat_theory}. On the other hand,
the ZIT assumes that agents instead of being fully rational, have ``zero
intelligence” and randomly buy or sell. It is supposed that their actions are
interpreted by other agents as potentially containing some information
\cite{subtle_nature,Bouchaud_2004,stat_theory,Wang_2016_cross}.
In both cases the outcome is the same, the prices follow a random walk. Reality
is somewhere in-between \cite{Bouchaud_2004,stat_theory}, and non-Markovian
effects due to stategies or liquidity costs are not contained either.

There are diverse studies focused on the price response
\cite{prop_order_book,dissecting_cross,r_walks_liquidity,subtle_nature,Bouchaud_2004,large_prices_changes,pow_law_dist,theory_market_impact,spread_changes_affect,master_curve,EMH_lillo,quant_stock_price_response,ori_pow_law,Wang_2018_b,Wang_2018_a,Wang_2016_avg,Wang_2016_cross},
In our opinion, a critical investigation of definitions and methods and how
they affect the results is called for.

Regarding price self-response functions in Refs.
\cite{r_walks_liquidity,subtle_nature,Bouchaud_2004}, Bouchaud et al. found an
increase to a maximum followed by a decrease as the time lag grows.
In Ref. \cite{theory_market_impact}, Gerig found that larger sized transactions
have a larger absolute impact than smaller sized transactions but a much
smaller relative impact. In Ref. \cite{prop_order_book}, it is found that the
impact of small trades on the price is, in relative terms, much larger than
that of large trades and the impact of trading on the price is quasi-permanent.

For price cross-responses functions, Refs.
\cite{dissecting_cross,Wang_2016_cross} revealed that the diagonal terms are on
average larger than the off-diagonal ones by a factor $\sim 5$. The response at
positive times is roughly constant, what is consistent with the hypothesis of a
statistically efficient price. Thus, the current sign does not predict future
returns. In Ref. \cite{Wang_2016_cross} the trends in the cross-responses were
found not to depend on whether or not the stock pairs are in the same economic
sector or extend over two sectors.

Here, we want to discuss, based on a series of detailed empirical results
obtained on trade by trade data, that the variation in the details of the
parameters used in the price response definition modify the characteristics of
the results. Aspects like time scale, time shift, time lag and spread used in
the price response calculation have an influence on the outcomes. To facilitate
the reproduction of our results, the source code for the data analysis is
available in Ref. PONER REFERENCIA.

We delve into the key details needed to compute the price response functions,
and explore their corresponding roles. We perform a empirical study in
different time scales. We show that the order between the trade signs and
the returns have a key importance in the price response signal. We split the
time lag to understand the contribution of the immediate returns and the late
returns. Finally, we shed light on the spread impact in the response functions
for single stocks.

The paper is organized as follows: in Sect. \ref{sec:data} we present our data
set of stocks. We then analyze the definition of the price response functions
and describe the physical and trade time in Sect.
\ref{sec:response_functions_def}. We implement different price responses for
several stocks and pairs of stocks in Sect. \ref{sec:response_functions_imp}.
In Sect. \ref{sec:time_shift} we show how the relative position between trade
signs and returns has a huge influence in the results of the computation of the
response functions. In Sect. \ref{sec:short_long} we explain in detail how the
time lag $\tau$ behaves in the response functions. Finally, in Sect.
\ref{sec:spread_impact} we analyze the spread impact in the price response
functions. Our conclusions follow in Sect. \ref{sec:conclusion}.