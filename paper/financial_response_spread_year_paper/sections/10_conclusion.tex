\section{Conclusion}\label{sec:conclusion}

We went into detail about the response functions in correlated financial
markets. We define the trade time scale and physical time scale to compute the
self- and cross-response functions for six companies with the largest average
market capitalization for three different economic sectors of the S\&P index
in 2008.
Due to the characteristics of the data used, we had to classify and sampling
values to obtain the corresponding quantities in different time scales.
The classification and sampling of the data had impact on the results, making
them smoother or stronger, but always keeping their shape and behavior.

The response functions were analyzed according to the time scales. We proposed
a new approach to compare price response functions from different scales. We
used the same midpoint prices in physical time scale with the corresponding
trade signs in trade time scale or physical time scale. This assumption allowed
us to compare both price response functions and get an idea of how
representative was the behavior obtained in both cases. For trade time scale,
the signal is weaker due to the large averaging values from all the trades in a
year. In the physical time scale, the response functions had less noise and
their signal were stronger. We proposed an activity response to measure how the
number of trades in every second highly impact the responses. As the response
functions can not grow indefinitely with the time lag, they increase to a peak,
to then decrease. It can be seen that the market needs time to react and revert
the growing. In both time scale cases depending on the stocks, two
characteristics behavior were shown. In one, the time lag was large enough to
show the complete increase-decrease behavior. In the other case, the time lag
was not enough, so some stocks only showed the growing behavior.

We modify the response function to add a time shift parameter. With this
parameter we wanted to analyze the importance in the order of the relation
between returns and trade signs. In trade time scale and physical time scale we
found similar results. When we shift the order between returns and trade signs,
the information from the relation between them is temporarily lost and as
outcome the signal does not have any meaningful information. When the order is
recovered, the response function grows again, showing the expected shape.
We showed that this is not an isolated conduct, and that all the shares used in
our analysis exhibit the same behavior. Thus, even if they are values of time
shift that can give a response function signal, empirically we propose this
time shift should be a value between $t_{s} = \left(0,2\right]$ time steps.

We analyzed the impact of the time lag in the response functions. We divided
the time lag in a short and long time lag. With this division we adapted the
price response function in physical time scale. The response function that
depended on the short time lag, showed a stronger response. The long response
function vanish, and depending on the stock could take negative and
non-negative values comparable to a random signal.

Finally, we checked the spread impact in price self-response functions. We
divided 524 stocks from the NASDAQ stock market in three groups depending on
the year average spread of every stock. The response functions signal were
stronger for the group of stocks with the larger spreads and weaker for the
group of stocks with the smaller spreads. A general average price response
behavior was spotted for the three groups, suggesting a market effect on the
stocks.