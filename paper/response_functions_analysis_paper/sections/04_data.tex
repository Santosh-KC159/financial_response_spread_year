\section{Data set and time definition}\label{sec:data_time}

In Sect. \ref{subsec:data_set} we introduce the data set used in the paper. In
Sect. \ref{subsec:time_definition} we describe the physical time scale and the
trade time scale.

%%%%%%%%%%%%%%%%%%%%%%%%%%%%%%%%%%%%%%%%%%%%%%%%%%%%%%%%%%%%%%%%%%%%%%%%%%%%%%%
\subsection{Data set}\label{subsec:data_set}

In this study, we have analyzed trades and quotes (TAQ) data from the NASDAQ
Stock Market. We selected NASDAQ because it is an electronic exchange where
stocks are traded through an automated network of computers instead of a
trading floor, which makes trading more efficient, fast and accurate.
NASDAQ is the second largest stock exchange based on market capitalization in
the world.

In the TAQ data set, there are two data files for each stock. One gives the
list of all successive quotes. Thus, we have the best bid price, best ask
price, available volume and the time stamp accurate to the second. The other
data file is the list of all successive trades, with the traded price, traded
volume and time stamp accurate to the second. Despite the one second accuracy
of the time stamps, in both files more than one quote or trade may be recorded
in the same second.

Due to the the time stamp accuracy, it is not possible to match each trade with
the directly preceding quote. Hence, we cannot determine the trade sign by
comparing the traded price and the preceding midpoint price
\cite{Wang_2016_cross}. In this case we need to do a preprocessing of the data
to relate the midpoint prices with the trade signs in trade time scale and in
physical time scale.

To analyze the response functions across different liquid stocks in Sects.
\ref{sec:response_functions}, \ref{sec:time_shift} and \ref{sec:short_long},
we select the six companies with the largest average market capitalization
(AMC) (Alphabet Inc., Mastercard Inc., CME Group Inc., Goldman Sachs Group
Inc., Transocean Ltd. and Apache Corp.) in three economic sectors (information
technology, financials and energy) of the S\&P index in 2008.

\begin{table*}[htbp]
\begin{threeparttable}
\caption{Analyzed companies.}
\begin{tabular*}{\textwidth}{c @{\extracolsep{\fill}} ccccc}
\toprule
\bf{Company} & \bf{Symbol} & \bf{Sector} & \bf{Quotes}\tnote{1} &
\bf{Trades}\tnote{2} & \bf{Spread}\tnote{3}\tabularnewline
\midrule
Alphabet Inc. & GOOG & Information Technology (IT) & $164489$ & $19029$ &
$\$0.04$\tabularnewline
Mastercard Inc. & MA & Information Technology (IT) & $98909$ & $6977$ &
$\$0.38$\tabularnewline
CME Group Inc. & CME & Financials (F) & $98188$ & $3032$ &
$\$1.08$\tabularnewline
Goldman Sachs Group Inc. & GS & Financials (F) & $160470$ & $26227$ &
$\$0.11$\tabularnewline
Transocean Ltd. & RIG & Energy (E) & $107092$ & $11641$ &
$\$0.12$\tabularnewline
Apache Corp. & APA & Energy (E) & $103074$ & $8889$ & $\$0.13$\tabularnewline
\bottomrule
\end{tabular*}
\label{tab:companies}
\begin{tablenotes}\footnotesize
\item[1] Average number of quotes from 9:40:00 to 15:50:00 New York time.
\item[2] Average number of trades from 9:40:00 to 15:50:00 New York time.
\item[3] Average spread from 9:40:00 to 15:50:00 New York time.
\end{tablenotes}
\end{threeparttable}
\end{table*}

Table \ref{tab:companies} shows the companies analyzed with their corresponding
symbol and sector. The highest average number of quotes per day and the most
liquid stock on average from our selection for the year 2008 is Alphabet Inc.
The most traded stock on average from the group was Goldman Sachs Group Inc.
On the other side, the stock with the less quotes, less traded and less
liquidity on average for the analyzed year was CME Group Inc.

To analyze the spread impact in response functions (Sect.
\ref{sec:spread_impact}), we select 530 stocks in the NASDAQ stock market for
the year 2008. The selected stocks are listed in Appendix
\ref{app:spread_impact}.

In order to avoid overnight effects and any artifact due to the opening and
closing of the market, we systematically discarded the first ten and the last
ten minutes of trading in a given day
\cite{Bouchaud_2004,large_prices_changes,spread_changes_affect,Wang_2016_cross}.
Therefore, we only consider trades of the same day from 9:40:00 to 15:50:00
New York local time. We will refer to this interval of time as the ``market
time".

%%%%%%%%%%%%%%%%%%%%%%%%%%%%%%%%%%%%%%%%%%%%%%%%%%%%%%%%%%%%%%%%%%%%%%%%%%%%%%%

In a modern financial market, there is a double continuous auction. To find
possible buyers and sellers in the market, agents can place different types of
instructions (known as orders) to buy or to sell a given number of shares, that
can be grouped into two categories: market orders and limit orders.

Market orders will go into market to execute at the best available buy or sell
price, they are executed as fast as possible and only after the purchase of the
stock is possible to know the exact price \cite{large_prices_changes,predictive_pow}.

Limit orders allow to set a maximum purchase price for a buy order, or a
minimum sale price for a sell order. If the market does not reach the limit
price, the order will not be executed \cite{large_prices_changes,predictive_pow}.

Limit orders often fail to result in an immediate transaction, and are stored
in a queue called the limit order book \cite{prop_order_book,stat_prop,predictive_pow}. An order book is an electronic list of
buy and sell orders for a specific security or financial instrument organized
by price level. An order book lists the number of shares being bid or offered
at each price point. It also identifies the market participants behind the buy
and sell orders, although some choose to remain anonymous. The order book is
visible for all traders, its main purpose is to ensure that all traders have
the information about what is offered on the market.

Buy limit orders are called ``bids", and sell limit orders are called ``asks".
At any given time there is a best (lowest) offer to sell with price
$a\left(t\right)$, and a best (highest) bid to buy with price
$b\left(t\right)$ \cite{subtle_nature,account_spread,limit_ord_spread,prop_order_book}. These
are also called the inside quotes or the best prices. The price gap between
them is called the spread $s\left(t\right) = a\left(t\right)-b\left(t\right)$
\cite{subtle_nature,Bouchaud_2004,large_prices_changes,account_spread,market_digest}.
Spreads are significantly positively related to price and significantly
negatively related to trading volume. Firms with more liquidity tend to have
lower spreads.
\cite{account_spread,effects_spread,components_spread,components_spread_tokyo}.