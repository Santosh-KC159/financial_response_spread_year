\section{Introduction}\label{sec:introduction}

reference EMH - \cite{EMH_lillo}

In this paper, we want to discuss, based on a series of detailed empirical
results obtained on trade by trade data, that the variation in the details of
the parameters used in the price response definition modify the characteristics
of the results. Aspects like time scale, time shift, time lag and spread used
in the price response calculation have a large impact on the outcomes.

There are diverse studies focused on the price response \cite{EMH_lillo,Bouchaud_2004,Wang_2016_cross,Wang_2016_avg,Wang_2018_a,Wang_2018_b,large_prices_changes,theory_market_impact,dissecting_cross,spread_changes_affect,quant_stock_price_response,master_curve,ori_pow_law,pow_law_dist,prop_order_book},
but they concentrate on a general definition and discuss the results without
going deep in the specific details of the price response measurements.

Here, we delve into the key details needed to compute the price response
functions, and investigate their corresponding roles. We perform a empirical
study in different time scales using financial data and find that using
different implementations of the price response, the results are qualitatively
the same, but the response can vary up to a factor of two. We show that the
order between the trade signs and the returns have a key importance in the
price response signal and suggest an interval where the time shift have to be
set. We split the time lag to understand the contribution of the immediate
returns and the late returns. The price response is highly influenced by the
instantaneous returns and as the time lag grows, the influence starts to
decrease. We shed light on the spread impact in the response functions for
single stocks. We prove that when the spread is large, the price response tend
to be large.



Acá empieza ...

The difference between the definition in \cite{Wang_2016_cross} and in
\cite{Wang_2018_b}, is that \cite{Wang_2016_cross} measures how a buy or sell
order at time $t$ influences on average the price at a later time $t + \tau$.
The physical time scale was chosen since the trades in different stocks are not
synchronous (TAQ data). In \cite{Wang_2018_b}, it was used a response function
on a trade time scale (Totalview data), as the interest is to analyze the
immediate responses. In \cite{Wang_2018_b} the time lag $\tau$ is restricted to
one, such that the price response quantifies the price impact of a single
trade.

The paper is organized as follows: in Sect. \ref{sec:data_time} we present our
data set of stocks and describe the physical and trade time. We then
analyze the definition of the response functions in Sect.
\ref{sec:response_functions}, and compute them for several stocks and pairs of
stocks. In Sect. \ref{sec:time_shift} we show how the relative position between
trade signs and returns has a huge impact in the results of the computation of
the response functions. In Sect. \ref{sec:short_long} we explain in detail how
the time lag $\tau$ behaves in the response functions. Finally, in Sect.
\ref{sec:spread_impact} analyze the spread impact in the response functions.
 Our conclusions follows in Sect. \ref{sec:conclusion}.