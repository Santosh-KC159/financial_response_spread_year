\section{Introduction}\label{sec:introduction}

Shares are the equal parts in which the capital of a company is divided.
A share is an indivisible unit of capital, expressing the ownership
relationship betwen a company and a shareholder. The shareholder owns a
percentage of the company depending on the amount of shares he has.
Shares are issued in two moments: when the company is created and when the
company want to raise funds.

The shares can be taken as an invesment, and receive dividends from them, or
they can be traded at any time.  This second possibility is the want that
concerns us.

To trade the stocks exist markets where the buyers and sellers meet. sellers
transfer (in exchange for money) the ownership of equities to buyers. This
requires these two parties to agree on a price.

In a modern financial market, there is a double continous auction. To find
possible buyers and sellers in the market, agents can place different types of
instructions (known as orders) to buy or to sell a given number of shares, that
can be grouped into two categories: market orders and limit orders.

Market orders will go into market to execute at the best available buy or sell
price, they are executed as fast as possible and only after the purchase of the
stock is possible to know the exact price \cite{large_prices_changes}.

Limit orders allow to set a maximum purchase price for a buy order, or a
minimum sale price for a sell order. If the market does not reach the limit
price, the order will not be executed \cite{large_prices_changes}.

Limit orders often fail to result in an immediate transaction, and are stored
in a queue called the limit order book. An order book is an electronic list of
buy and sell orders for a specific security or financial instrument organized
by price level. An order book lists the number of shares being bid or offered
at each price point. It also identifies the market participants behind the buy
and sell orders, although some choose to remain anonymous. The order book is
visible for all traders, its main purpose is to ensure that all traders have
the information about what is offered on the market.

Buy limit orders are called ``bids", and sell limit orders are called ``asks".
At any given time there is a best (lowest) offer to sell with price
$a\left(t\right)$, and a best (highest) bid to buy with price
$b\left(t\right)$ \cite{subtle_nature}. These are also called the inside quotes
or the best prices. The price gap between them is called the spread
$s\left(t\right) = a\left(t\right)-b\left(t\right)$
\cite{subtle_nature,Bouchaud_2004,large_prices_changes}.

The average of the best ask and the best bid is the midpoint price, which is
defined as \cite{subtle_nature,Bouchaud_2004,large_prices_changes}

\begin{equation}\label{eq:midpoint_price}
    m\left(t\right)=\frac{a\left(t\right)+b\left(t\right)}{2}
\end{equation}

As the midpoint price depends on the quotes, it changes if the quotes change.
The midpoint price grows if the best ask or the best bid grow. This happen if
someone buys and consumes all the volume of the sell limit order with the price
of the best ask, or someone sets a buy limit order with a bigger price than the
previous best bid, or there is a cancellation of the best ask.

On the other hand, the midpoint price decreases if the best ask or the best bid
decrease. This happen if someone sells and consumes all the volume of the buy
limit order with the price of the best bid, or someone sets a sell limit order
with a lower price than the previous best bid, or there is a cancellation of
the best bid.

The midpoint price will not change if there is no activity in the market.

Price changes are typically characterized as returns. If one denotes
$S\left( t\right)$ the price of an asset at time $t$, the return
$r\left(t\right)$, at time $t$ and time lag $\tau$ is simply the relative
variation of the price from $t$ to $t + \tau$
\cite{subtle_nature,empirical_facts},

\begin{equation}\label{eq:return_general}
    r^{g} \left(t, \tau \right) = \frac{S\left(t + \tau\right)
    - S\left(t\right)}{S\left(t\right)}
\end{equation}

It is also common to define the returns as
\cite{dissecting_cross,subtle_nature,empirical_facts,empirical_properties,large_prices_changes,theory_market_impact,spread_changes_affect,fluctions_market_friction}

\begin{equation}\label{eq:log_return_general}
    r^{l}\left(t,\tau\right) = \ln S\left(t + \tau\right)
    - \ln S\left(t\right) = \ln \frac{S\left(t + \tau\right)}{S\left(t\right)}
\end{equation}

Equation \ref{eq:return_general} and Eq. \ref{eq:log_return_general} coincide
if $\tau$ is small enough \cite{subtle_nature,empirical_facts}.

At longer timescales, midpoint prices and transaction prices rarely differ by
more than half the spread. The midpoint price is more convenient to study
because it avoids problems associated with the tendency of transaction prices
to bounce back and forth between the best bid and ask
\cite{large_prices_changes}.

We define the returns via the midpoint price as

\begin{equation}\label{eq:midpoint_price_return}
    r\left(t,\tau\right) = \frac{m\left(t+\tau\right)-m\left(t\right)}
    {m\left(t\right)}
\end{equation}

The distribution of returns is strongly non-Gaussian and its shape continuously
depends on the return period $\tau$. Small $\tau$ values have fat tails return
distributions \cite{subtle_nature}.

Then we can expect three kind of values of the returns. The returns are
positive values, when the midpoint price
$m\left(t+\tau\right) > m\left(t\right)$, hence, there is a buy in the market
or there is a cancellation of the best ask or an addition in the best bid
during the time lag $\tau$. The returns are negative values, when the midpoint
price $m\left(t+\tau\right) < m\left(t\right)$, thus, there is a sell in the
market, or there is a cancellation of the best bid or an addition in the best
ask during the time lag $\tau$. The returns are zero when there is no activity
during the time lag $\tau$.

The trade signs are defined for general cases as

\begin{equation}\label{eq:trade_sign_general}
    \varepsilon\left(t\right)=\text{sign}\left(S\left(t\right)
    -m\left(t-\delta\right)\right)
\end{equation}

where $\delta$ is a positive time increment. Hence we have

\begin{equation}\label{eq:trade_sign_results}
    \varepsilon\left(t\right)=\left\{
    \begin{array}{cc}
    +1, & \text{If } S\left(t\right)
    \text{ is higher than the last } m\left( t \right)\\
    -1, & \text{If } S\left(t\right)
    \text{ is lower than the last } m\left( t \right)
    \end{array}\right.
\end{equation}

$\varepsilon(t) = +1$ indicates that the trade was triggered by a market order
to buy and a trade triggered by a market order to sell yields
$\varepsilon(t) = -1$
\cite{subtle_nature,Bouchaud_2004,spread_changes_affect,quant_stock_price_response,order_flow_persistent}.

It is well-known that the series of the trade signs on a given stock exhibit
large autocorrelation. A very plausible explanation of this phenomenon relies
on the execution strategies of some major brokers on a given markets. These
brokers have large transaction to execute on the account of some clients. In
order to avoid market making move because of an inconsiderable large order,
they tend to split large orders into small ones \cite{empirical_facts}.

The response function is used to study the mutual dependence between stocks. In
\cite{r_walks_liquidity,Bouchaud_2004}, Bouchaud et al. use the response
function that only depends on the time lag $\tau$

\begin{equation}\label{eq:Bouchaud_2004}
    R\left(\tau\right)=\left\langle \left(S_{n+\tau}-S_{n}\right) \cdot
    \varepsilon_{n}\right\rangle_{trades}
\end{equation}

Where $\varepsilon_{n}$ is the sign of the $n^{th}$ trade and the price $S_n$
is defined as the midpoint just before the $n^{th}$ trade
($S_{n} \equiv m_{n^{-}}$).
The quantity $R\left(\tau\right)$ measures how much, on average, the price
moves up (down) at time $\tau$ conditioned to a buy (sell) order at time zero.

In a later work \cite{Wang_2016_cross}, S. Wang et al. use the logarithmic
return for stock $i$ and time lag $\tau$, defined via the midpoint price
$m_{i} \left( t \right)$. The cross-response function is then defined as

\begin{equation}\label{eq:Wang_2016}
    R_{ij}\left(\tau\right)=\left\langle r_{i}\left(t-1,\tau\right)\cdot
    \varepsilon_{j} \left(t\right) \right\rangle _{t}
\end{equation}

Finally, in \cite{Wang_2018_b}, S. Wang et al. define the response function as

\begin{equation}\label{eq:Wang_2018_b}
    R_{ij} = \left\langle \left(\ln m_{i}^{\left(f\right)}\left(t_{j}\right)-
    \ln m_{i}^{\left(p\right)} \left(t_{j}\right) \right)\cdot\varepsilon_{j}
    \left(t_{j}\right)\right\rangle _{t_{j}}
\end{equation}

For the price change of stock $i$ caused by a trade of stock $j$.

Here, $m_{i}^{\left(p\right)}\left(t_{j}\right)$ is the midpoint price of stock
$i$ previous to the trade of stock $j$ at its event time $t_j$ and
$m_{i}^{\left(f\right)}\left(t_{j}\right)$ is the midpoint price of stock $i$
following that trade.

The difference between the definition in \cite{Wang_2016_cross} and in
\cite{Wang_2018_b}, is that \cite{Wang_2016_cross} measures how a buy or sell
order at time $t$ influences on average the price at a later time $t + \tau$.
The physical time scale was chosen since the trades in different stocks are not
synchronous (TAQ data). In \cite{Wang_2018_b}, it was used a response function
on a trade time scale (Totalview data), as the interest is to analyze the
immediate responses. In \cite{Wang_2018_b} the time lag $\tau$ is restricted to
one, such that the price response quantifies the price impact of a single
trade.

The paper is organized as follows: in Sect. \ref{sec:data_time} we present our
data set of stocks and describe the physical and trade time. We then
analyze the definition of the response functions in Sect.
\ref{sec:response_functions}, and compute them for several stocks and pairs of
stocks. In Sect. \ref{sec:time_shift} we show how the relative position between
trade signs and returns has a huge impact in the results of the computation of
the response functions. Finally, in Sect. \ref{sec:short_long} we explain in
detail how the time lag $\tau$ behaves in the response functions. Our
conclusions follows in Sect. \ref{sec:conclusion}.