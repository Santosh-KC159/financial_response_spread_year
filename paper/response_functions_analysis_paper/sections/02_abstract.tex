\abstract{
Using Trades and Quotes (TAQ) data from the NASDAQ stock market, we analyzed
the response functions of six companies with the largest average market
capitalization in three economic sector of the S\&P index in 2008. We used two
time definitions to compute the response functions: trade time scale and
physical time scale. We computed the self-response functions for the six
companies and the cross-response functions for the three sectors. Both increase
to a maximum and then slowly decrease. Hence, the trend in the response
functions is eventually reversed. To analyze the influence of the relative
position between trade signs and returns, we added a parameter time shift
$t_{s}$ to the response function expression. For negative and large values of
the time shift the information contained between the trade signs and returns
is lost. Finally, we analyzed the impact of the time lag in the response
functions. We divided the time lag in an immediate and late component and
compute the self- and cross-response. The influence of the immediate time lag
is larger than the late time lag.
\PACS{
      {89.65.Gh}{Econophysics} \and
      {89.75.-k}{Complex systems} \and
      {05.10.Gg}{Statistical physics}
     } % end of PACS codes
} %end of abstract